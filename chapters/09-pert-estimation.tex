% ============================================================================
% CHAPTER 8: PERT TIME ESTIMATION
% ============================================================================

\chapter{PERT Time Estimation}

\vspace{12pt}

\noindent This chapter presents the Program Evaluation and Review Technique (PERT) time estimates for project activities. PERT uses three-point estimation to account for uncertainty in activity durations.

% ============================================================================
\section{PERT Overview}

\par PERT is a statistical tool that uses three time estimates for each activity:

\begin{itemize}[leftmargin=*]
    \item \textbf{Optimistic Time (O):} Minimum time if everything goes perfectly
    \item \textbf{Most Likely Time (M):} Most realistic time estimate
    \item \textbf{Pessimistic Time (P):} Maximum time if significant problems occur
\end{itemize}

\subsection{PERT Formulas}

\par \textbf{Expected Time (TE):}
\begin{equation}
TE = \frac{O + 4M + P}{6}
\end{equation}

\par \textbf{Standard Deviation (σ):}
\begin{equation}
\sigma = \frac{P - O}{6}
\end{equation}

\par \textbf{Variance (σ²):}
\begin{equation}
\sigma^2 = \left(\frac{P - O}{6}\right)^2
\end{equation}

% ============================================================================
\section{PERT Time Estimates for All Activities}

\begin{landscape}

\begin{table}[H]
\centering
\caption{PERT Three-Point Time Estimates}
\small
\begin{tabularx}{\linewidth}{|l|X|c|c|c|c|c|c|}
\hline
\textbf{ID} & \textbf{Activity Name} & \textbf{O} & \textbf{M} & \textbf{P} & \textbf{TE} & \textbf{σ} & \textbf{σ²} \\
 & & \textbf{(weeks)} & \textbf{(weeks)} & \textbf{(weeks)} & \textbf{(weeks)} & & \\
\hline
A & Project Initiation & 1.5 & 2.0 & 3.0 & 2.08 & 0.25 & 0.06 \\
\hline
B & Requirements Gathering & 3.0 & 4.0 & 6.0 & 4.17 & 0.50 & 0.25 \\
\hline
C & Requirements Documentation & 1.5 & 2.0 & 3.0 & 2.08 & 0.25 & 0.06 \\
\hline
D & System Architecture Design & 2.0 & 3.0 & 5.0 & 3.17 & 0.50 & 0.25 \\
\hline
E & Database Design & 1.5 & 2.0 & 3.0 & 2.08 & 0.25 & 0.06 \\
\hline
F & UI/UX Design & 2.0 & 3.0 & 5.0 & 3.17 & 0.50 & 0.25 \\
\hline
G & Backend Development Setup & 0.5 & 1.0 & 1.5 & 1.00 & 0.17 & 0.03 \\
\hline
H & Prerequisite Engine Development & 2.0 & 3.0 & 5.0 & 3.17 & 0.50 & 0.25 \\
\hline
I & Conflict Detection Module & 1.5 & 2.0 & 3.0 & 2.08 & 0.25 & 0.06 \\
\hline
J & Waitlist Management System & 1.5 & 2.0 & 3.0 & 2.08 & 0.25 & 0.06 \\
\hline
K & Registration Workflow Implementation & 2.0 & 3.0 & 5.0 & 3.17 & 0.50 & 0.25 \\
\hline
L & Frontend Development Setup & 0.5 & 1.0 & 1.5 & 1.00 & 0.17 & 0.03 \\
\hline
M & Student Portal Development & 3.0 & 4.0 & 6.0 & 4.17 & 0.50 & 0.25 \\
\hline
N & Admin Dashboard Development & 2.0 & 3.0 & 5.0 & 3.17 & 0.50 & 0.25 \\
\hline
O & Course Catalog Interface & 1.5 & 2.0 & 3.0 & 2.08 & 0.25 & 0.06 \\
\hline
P & Timetable Builder Component & 2.0 & 3.0 & 5.0 & 3.17 & 0.50 & 0.25 \\
\hline
Q & Frontend-Backend Integration & 1.5 & 2.0 & 3.0 & 2.08 & 0.25 & 0.06 \\
\hline
R & SIS Integration & 1.5 & 2.0 & 4.0 & 2.25 & 0.42 & 0.17 \\
\hline
S & Authentication Integration & 0.5 & 1.0 & 2.0 & 1.08 & 0.25 & 0.06 \\
\hline
T & Email Notification System & 0.5 & 1.0 & 2.0 & 1.08 & 0.25 & 0.06 \\
\hline
U & Unit Testing & 1.5 & 2.0 & 3.0 & 2.08 & 0.25 & 0.06 \\
\hline
V & Integration Testing & 1.5 & 2.0 & 3.0 & 2.08 & 0.25 & 0.06 \\
\hline
W & System Testing & 1.5 & 2.0 & 3.0 & 2.08 & 0.25 & 0.06 \\
\hline
X & Performance Testing & 0.5 & 1.0 & 2.0 & 1.08 & 0.25 & 0.06 \\
\hline
Y & Security Testing & 0.5 & 1.0 & 2.0 & 1.08 & 0.25 & 0.06 \\
\hline
Z & User Acceptance Testing & 1.5 & 2.0 & 3.0 & 2.08 & 0.25 & 0.06 \\
\hline
AA & Deployment Planning & 0.5 & 1.0 & 1.5 & 1.00 & 0.17 & 0.03 \\
\hline
AB & Production Deployment & 0.5 & 1.0 & 2.0 & 1.08 & 0.25 & 0.06 \\
\hline
AC & Training Material Development & 1.5 & 2.0 & 3.0 & 2.08 & 0.25 & 0.06 \\
\hline
AD & Training Delivery & 0.5 & 1.0 & 1.5 & 1.00 & 0.17 & 0.03 \\
\hline
AE & Go-Live Support & 1.5 & 2.0 & 3.0 & 2.08 & 0.25 & 0.06 \\
\hline
AF & Project Closure & 0.5 & 1.0 & 1.5 & 1.00 & 0.17 & 0.03 \\
\hline
\end{tabularx}
\end{table}

\end{landscape}

% ============================================================================
\section{Critical Path PERT Analysis}

\par Calculating the expected duration and variance for the critical path:

\begin{table}[H]
\centering
\caption{Critical Path PERT Summary}
\begin{tabular}{|l|c|c|}
\hline
\textbf{Metric} & \textbf{Value} & \textbf{Unit} \\
\hline
Sum of Expected Times (TE) & 34.75 & weeks \\
\hline
Sum of Variances (Σσ²) & 1.79 & weeks² \\
\hline
Critical Path Standard Deviation & 1.34 & weeks \\
\hline
\end{tabular}
\end{table}

% ============================================================================
\section{Project Duration Probability Analysis}

\par Using the PERT analysis, we can calculate probabilities for project completion times.

\subsection{Z-Score Calculations}

\par For a target completion time (T), the Z-score is:
\begin{equation}
Z = \frac{T - TE_{total}}{\sigma_{critical path}}
\end{equation}

\subsection{Probability Scenarios}

\begin{table}[H]
\centering
\caption{Probability of Completing Project by Target Date}
\begin{tabular}{|c|c|c|c|}
\hline
\textbf{Target Duration} & \textbf{Z-Score} & \textbf{Probability} & \textbf{Confidence} \\
\textbf{(weeks)} & & & \\
\hline
33 & -1.31 & 9.5\% & Very Low \\
\hline
34 & -0.56 & 28.8\% & Low \\
\hline
35 & 0.19 & 57.5\% & Moderate \\
\hline
36 & 0.93 & 82.4\% & High \\
\hline
37 & 1.68 & 95.4\% & Very High \\
\hline
38 & 2.43 & 99.2\% & Near Certain \\
\hline
\end{tabular}
\end{table}

\subsection{Interpretation}

\begin{itemize}[leftmargin=*]
    \item There is approximately 57.5\% probability of completing in 35 weeks
    \item For 82.4\% confidence, allow 36 weeks (2 weeks buffer)
    \item For 95\% confidence level, plan for 37 weeks completion
    \item Original estimate of 34 weeks has only 28.8\% probability of success
\end{itemize}

% ============================================================================
\section{Recommendations}

\par Based on PERT analysis:

\begin{enumerate}[leftmargin=*]
    \item \textbf{Add Buffer Time:} Include 2-3 weeks buffer for high confidence
    \item \textbf{Monitor High-Variance Activities:} Focus on activities with σ > 0.40
    \item \textbf{Early Start Critical Activities:} Begin prerequisite engine and requirements early
    \item \textbf{Resource Backup Plans:} Have contingency for integration activities
    \item \textbf{Regular Re-estimation:} Update PERT estimates as project progresses
\end{enumerate}

% ============================================================================
\section{Risk-Based Schedule Buffer}

\begin{table}[H]
\centering
\caption{Recommended Schedule Buffers}
\begin{tabular}{|l|c|l|}
\hline
\textbf{Confidence Level} & \textbf{Buffer} & \textbf{Rationale} \\
\hline
70\% & 1 week & Minimum acceptable buffer \\
\hline
80\% & 2 weeks & Recommended for this project \\
\hline
90\% & 2.5 weeks & Conservative estimate \\
\hline
95\% & 3 weeks & Very conservative, high assurance \\
\hline
\end{tabular}
\end{table}

\par \textbf{Recommendation:} Adopt a 36-week project timeline (34.75 + 2 weeks buffer) for approximately 82\% confidence in meeting the deadline.
