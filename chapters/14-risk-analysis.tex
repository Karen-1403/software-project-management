% ============================================================================
% CHAPTER 13: RISK ANALYSIS
% ============================================================================

\chapter{Risk Analysis}

\vspace{12pt}

\noindent This chapter presents a comprehensive risk analysis for the Online Course Registration Portal project, including risk identification, assessment, probability-impact matrix, and mitigation strategies.

% ============================================================================
\section{Risk Management Overview}

\par Risk management is a systematic process of identifying, analyzing, and responding to project risks to increase the probability of positive events and decrease the probability of negative events.

\subsection{Risk Management Process}

\begin{enumerate}[leftmargin=*]
    \item Risk Identification
    \item Risk Analysis (Qualitative and Quantitative)
    \item Risk Response Planning
    \item Risk Monitoring and Control
\end{enumerate}

% ============================================================================
\section{Risk Identification}

\par The following risks have been identified across various project categories:

\subsection{Technical Risks}

\begin{table}[H]
\centering
\caption{Technical Risks}
\small
\begin{tabularx}{\textwidth}{|c|X|c|c|}
\hline
\textbf{ID} & \textbf{Risk Description} & \textbf{Probability} & \textbf{Impact} \\
\hline
T1 & Integration with legacy student information system fails or is more complex than anticipated & High & High \\
\hline
T2 & Prerequisite validation logic contains errors leading to incorrect enrollments & Medium & High \\
\hline
T3 & System performance degrades under peak concurrent user load & Medium & High \\
\hline
T4 & Timetable conflict detection algorithm produces false positives/negatives & Medium & Medium \\
\hline
T5 & Security vulnerabilities discovered in production & Low & High \\
\hline
T6 & Database scalability issues as data volume grows & Low & Medium \\
\hline
T7 & Third-party API dependencies become unavailable & Low & High \\
\hline
\end{tabularx}
\end{table}

\subsection{Resource Risks}

\begin{table}[H]
\centering
\caption{Resource Risks}
\small
\begin{tabularx}{\textwidth}{|c|X|c|c|}
\hline
\textbf{ID} & \textbf{Risk Description} & \textbf{Probability} & \textbf{Impact} \\
\hline
R1 & Key developers leave project mid-development & Medium & High \\
\hline
R2 & Insufficient QA resources during testing phase & Medium & Medium \\
\hline
R3 & Stakeholders not available for requirements validation and UAT & Medium & Medium \\
\hline
R4 & Lack of expertise in specific required technologies & Low & Medium \\
\hline
R5 & DevOps engineer unavailable during critical deployment phase & Low & High \\
\hline
\end{tabularx}
\end{table}

\subsection{Schedule Risks}

\begin{table}[H]
\centering
\caption{Schedule Risks}
\small
\begin{tabularx}{\textwidth}{|c|X|c|c|}
\hline
\textbf{ID} & \textbf{Risk Description} & \textbf{Probability} & \textbf{Impact} \\
\hline
S1 & Requirements gathering takes longer than estimated due to stakeholder unavailability & High & Medium \\
\hline
S2 & Integration testing reveals major defects requiring significant rework & Medium & High \\
\hline
S3 & UAT approval delayed due to discovered issues & Medium & High \\
\hline
S4 & Deployment delayed due to production environment not ready & Low & High \\
\hline
S5 & Training schedule conflicts with academic calendar & Medium & Low \\
\hline
\end{tabularx}
\end{table}

\subsection{Budget/Cost Risks}

\begin{table}[H]
\centering
\caption{Budget and Cost Risks}
\small
\begin{tabularx}{\textwidth}{|c|X|c|c|}
\hline
\textbf{ID} & \textbf{Risk Description} & \textbf{Probability} & \textbf{Impact} \\
\hline
C1 & Project costs exceed budget due to scope creep & Medium & Medium \\
\hline
C2 & Additional licensing costs for required third-party software & Medium & Low \\
\hline
C3 & Infrastructure costs higher than estimated & Low & Medium \\
\hline
C4 & Extended timeline increases personnel costs & Medium & Medium \\
\hline
\end{tabularx}
\end{table}

\subsection{External Risks}

\begin{table}[H]
\centering
\caption{External Risks}
\small
\begin{tabularx}{\textwidth}{|c|X|c|c|}
\hline
\textbf{ID} & \textbf{Risk Description} & \textbf{Probability} & \textbf{Impact} \\
\hline
E1 & Changes in data protection regulations require system modifications & Low & Medium \\
\hline
E2 & University policy changes affect registration business rules & Medium & Medium \\
\hline
E3 & Vendor discontinues support for critical technology component & Low & High \\
\hline
E4 & Cyber-security threats or attacks during go-live period & Low & High \\
\hline
\end{tabularx}
\end{table}

\subsection{Organizational Risks}

\begin{table}[H]
\centering
\caption{Organizational Risks}
\small
\begin{tabularx}{\textwidth}{|c|X|c|c|}
\hline
\textbf{ID} & \textbf{Risk Description} & \textbf{Probability} & \textbf{Impact} \\
\hline
O1 & Resistance to change from staff accustomed to manual processes & High & Medium \\
\hline
O2 & Students experience difficulty adapting to new system & Medium & Medium \\
\hline
O3 & Insufficient executive support during implementation challenges & Low & High \\
\hline
O4 & Competing priorities divert attention from project & Medium & Medium \\
\hline
\end{tabularx}
\end{table}

% ============================================================================
\section{Risk Probability-Impact Matrix}

\subsection{Risk Rating Scale}

\begin{table}[H]
\centering
\caption{Probability and Impact Scales}
\begin{tabular}{|l|l|c|l|}
\hline
\multicolumn{2}{|l|}{\textbf{Probability}} & \multicolumn{2}{l|}{\textbf{Impact}} \\
\hline
Very Low & < 10\% & Very Low & Minimal impact on objectives \\
\hline
Low & 10-30\% & Low & Minor impact, easily managed \\
\hline
Medium & 30-50\% & Medium & Moderate impact, requires attention \\
\hline
High & 50-70\% & High & Significant impact, major concern \\
\hline
Very High & > 70\% & Very High & Severe impact, project success threatened \\
\hline
\end{tabular}
\end{table}

\subsection{Probability-Impact Matrix}

\begin{figure}[H]
    \centering
    % Insert risk matrix diagram
    % \includegraphics[width=0.9\textwidth]{images/risk-matrix.pdf}
    \caption{Risk Probability-Impact Matrix}
    \label{fig:risk-matrix}
\end{figure}

\par \textit{Note: Create a 5x5 risk matrix with:}
\begin{itemize}[leftmargin=*]
    \item Y-axis: Probability (Very Low to Very High)
    \item X-axis: Impact (Very Low to Very High)
    \item Color coding: Green (Low Risk), Yellow (Medium Risk), Orange (High Risk), Red (Critical Risk)
    \item Plot all identified risks on the matrix using their IDs
\end{itemize}

% ============================================================================
\section{Risk Prioritization}

\begin{table}[H]
\centering
\caption{High-Priority Risks Requiring Immediate Attention}
\begin{tabular}{|c|l|c|}
\hline
\textbf{Risk ID} & \textbf{Description} & \textbf{Risk Score} \\
\hline
T1 & SIS Integration complexity & 0.60 × 0.80 = 0.48 \\
\hline
T3 & Performance under load & 0.40 × 0.80 = 0.32 \\
\hline
S2 & Integration testing defects & 0.40 × 0.80 = 0.32 \\
\hline
S3 & UAT approval delays & 0.40 × 0.80 = 0.32 \\
\hline
R1 & Key developer departure & 0.40 × 0.80 = 0.32 \\
\hline
O1 & Resistance to change & 0.60 × 0.50 = 0.30 \\
\hline
\end{tabular}
\end{table}

% ============================================================================
\section{Risk Response Planning}

\subsection{High-Priority Risk Mitigation Strategies}

\begin{table}[H]
\centering
\caption{Risk Response Strategies}
\small
\begin{tabularx}{\textwidth}{|c|X|X|l|}
\hline
\textbf{ID} & \textbf{Mitigation Strategy} & \textbf{Contingency Plan} & \textbf{Owner} \\
\hline
T1 & Early prototype integration; dedicated integration specialist; extra time buffer & Use manual data transfer procedures temporarily & TL, SA \\
\hline
T3 & Load testing in early phases; scalable architecture; CDN implementation & Add server capacity; optimize code & SA, DO \\
\hline
S2 & Continuous integration testing; code reviews; automated testing & Extend testing phase; add QA resources & QA Lead \\
\hline
S3 & Early stakeholder engagement; pilot testing; clear UAT criteria & Fast-track critical fixes; phased rollout & PM, BA \\
\hline
R1 & Cross-training; documentation; knowledge sharing sessions; competitive compensation & Contract backup developers; redistribute work & PM, TL \\
\hline
O1 & Change management program; early user involvement; training; communication plan & Executive intervention; additional support & PM, TR \\
\hline
\end{tabularx}
\end{table}

% ============================================================================
\section{Risk Monitoring and Control}

\subsection{Risk Tracking}

\begin{itemize}[leftmargin=*]
    \item Monthly risk register review
    \item Weekly tracking of top 10 risks
    \item Risk status reporting in project status meetings
    \item Trigger conditions defined for each risk
    \item Risk owners assigned and accountable
\end{itemize}

\subsection{Risk Escalation Criteria}

\begin{enumerate}[leftmargin=*]
    \item Risk probability or impact increases significantly
    \item Risk mitigation strategy proves ineffective
    \item New high-impact risks emerge
    \item Multiple risks materialize simultaneously
    \item Risk threatens critical project objectives
\end{enumerate}

% ============================================================================
\section{Contingency Reserves}

\begin{table}[H]
\centering
\caption{Contingency Reserve Allocation}
\begin{tabular}{|l|r|l|}
\hline
\textbf{Reserve Type} & \textbf{Amount} & \textbf{Purpose} \\
\hline
Schedule Reserve & 2 weeks & Buffer for schedule risks \\
\hline
Budget Reserve & 145,000 EGP (10\%) & Cost overrun protection \\
\hline
Resource Reserve & 2 backup developers & Critical resource backup \\
\hline
\end{tabular}
\end{table}
