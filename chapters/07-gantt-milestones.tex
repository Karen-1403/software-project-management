% ============================================================================
% CHAPTER 6: GANTT CHART AND MILESTONES
% ============================================================================

\chapter{Gantt Chart and Milestones}

\vspace{12pt}

\noindent This chapter presents the project schedule in the form of a Gantt chart, showing all project activities, their durations, dependencies, and relationships. Additionally, all project milestones are identified and justified.

% ============================================================================
\section{Project Schedule Overview}

\par The project schedule spans [X] weeks/months from initiation to closure. The Gantt chart provides a visual timeline of all activities, showing start and end dates, durations, and dependencies between tasks.

\subsection{Schedule Development Approach}

\begin{itemize}[leftmargin=*]
    \item Activities derived from WBS work packages
    \item Durations estimated using expert judgment and historical data
    \item Dependencies identified through activity sequencing
    \item Resource constraints considered in scheduling
    \item Critical path analysis performed
    \item Schedule validated with stakeholders
\end{itemize}

% ============================================================================
\section{Gantt Chart}

\begin{landscape}

\begin{figure}[H]
    \centering
    % Insert Gantt chart created in MS Project
    % Export from MS Project as high-quality image or PDF
    % \includegraphics[width=\linewidth]{images/gantt-chart.pdf}
    \caption{Project Gantt Chart - Complete Schedule}
    \label{fig:gantt-chart}
\end{figure}

\par \textit{Note: Create a comprehensive Gantt chart in Microsoft Project showing:}
\begin{itemize}[leftmargin=*]
    \item All major project phases
    \item Key activities and work packages
    \item Task durations and dates
    \item Dependencies between activities
    \item Milestones marked with diamond symbols
    \item Critical path highlighted (typically in red)
    \item Resource assignments
    \item Current date indicator (baseline)
\end{itemize}

\end{landscape}

% ============================================================================
\section{Project Milestones}

\par Milestones are significant events or decision points in the project that have zero duration but mark important achievements. The following table lists all project milestones with their justifications.

\begin{table}[H]
\centering
\caption{Project Milestones and Justifications}
\begin{tabularx}{\textwidth}{|c|l|l|X|}
\hline
\textbf{No.} & \textbf{Milestone} & \textbf{Target Date} & \textbf{Justification} \\
\hline
M1 & Project Planning Approved & Week 4 & Ensures all stakeholders agree on project scope, timeline, and budget before commencing work. Provides authorization to proceed. \\
\hline
M2 & Requirements Approved & Week 8 & Confirms all functional and non-functional requirements are documented, validated, and approved. Prevents scope ambiguity in later phases. \\
\hline
M3 & Design Approved & Week 12 & Validates technical architecture and design decisions before development begins. Reduces rework and ensures technical feasibility. \\
\hline
M4 & Backend Development Complete & Week 18 & Marks completion of core business logic and services. Critical for frontend integration and system testing to proceed. \\
\hline
M5 & Frontend Development Complete & Week 21 & Signifies completion of user interface implementation. Enables end-to-end system integration and user testing. \\
\hline
M6 & Integration Complete & Week 23 & Confirms all system components work together seamlessly. Essential before comprehensive testing begins. \\
\hline
M7 & Development Phase Complete & Week 24 & Marks code freeze and transition to formal testing phase. All features implemented and ready for validation. \\
\hline
M8 & System Testing Complete & Week 27 & Indicates all functional and non-functional tests passed. System meets quality standards and is ready for user acceptance. \\
\hline
M9 & UAT Approved & Week 28 & Stakeholder validation that system meets business needs. Critical go/no-go decision point for deployment. \\
\hline
M10 & System Deployed & Week 30 & Production deployment successful and verified. System infrastructure operational and ready for users. \\
\hline
M11 & Training Complete & Week 31 & All users trained and prepared to use the system. Ensures smooth adoption and minimizes support issues. \\
\hline
M12 & System Go-Live & Week 32 & System officially operational for all students. Must occur before registration period begins. \\
\hline
M13 & First Registration Complete & Week 34 & Successful completion of first registration cycle. Validates system performance under real-world conditions. \\
\hline
M14 & Project Closed & Week 36 & All deliverables accepted, documentation complete, lessons learned captured. Formal project closure. \\
\hline
\end{tabularx}
\end{table}

% ============================================================================
\section{Milestone Importance and Impact}

\subsection{Critical Milestones}

\par The following milestones are critical to project success and have the highest priority:

\begin{enumerate}[leftmargin=*]
    \item \textbf{M3 - Design Approved:} Prevents costly rework during development by ensuring design correctness upfront
    \item \textbf{M9 - UAT Approved:} Final validation before production deployment; failure requires major remediation
    \item \textbf{M12 - System Go-Live:} Time-sensitive milestone aligned with academic calendar; delay impacts entire institution
\end{enumerate}

\subsection{Milestone Dependencies}

\par Each milestone has specific dependencies and enables subsequent project work:

\begin{table}[H]
\centering
\caption{Milestone Dependencies}
\begin{tabularx}{\textwidth}{|l|X|X|}
\hline
\textbf{Milestone} & \textbf{Prerequisites} & \textbf{Enables} \\
\hline
M1 & Stakeholder approval, resource allocation & Requirements gathering activities \\
\hline
M2 & Complete requirements documentation & Design activities, effort estimation \\
\hline
M3 & Approved requirements, design reviews & Development work to commence \\
\hline
M7 & All development work packages complete & Formal testing phase \\
\hline
M9 & Successful testing, stakeholder UAT & Production deployment authorization \\
\hline
M12 & Deployment, training, verification & Student registration activities \\
\hline
\end{tabularx}
\end{table}

% ============================================================================
\section{Schedule Baseline}

\par The approved project schedule serves as the baseline for performance measurement and control.

\subsection{Baseline Information}

\begin{itemize}[leftmargin=*]
    \item \textbf{Baseline Start Date:} [Insert Date]
    \item \textbf{Baseline Finish Date:} [Insert Date]
    \item \textbf{Total Duration:} [X] weeks
    \item \textbf{Number of Activities:} [X]
    \item \textbf{Number of Milestones:} 14
    \item \textbf{Baseline Approval Date:} [Insert Date]
    \item \textbf{Approved By:} Project Sponsor
\end{itemize}

\subsection{Schedule Performance Metrics}

\par The following metrics will be used to monitor schedule performance:

\begin{itemize}[leftmargin=*]
    \item \textbf{Schedule Variance (SV):} Earned Value - Planned Value
    \item \textbf{Schedule Performance Index (SPI):} Earned Value / Planned Value
    \item \textbf{Critical Path Duration:} Monitored weekly
    \item \textbf{Milestone Achievement Rate:} Percentage of milestones met on time
    \item \textbf{Task Completion Rate:} Percentage of tasks completed vs. planned
\end{itemize}

% ============================================================================
\section{Key Activity Durations Summary}

\begin{table}[H]
\centering
\caption{Major Phase Durations}
\begin{tabular}{|l|c|c|c|}
\hline
\textbf{Phase} & \textbf{Duration} & \textbf{Start Week} & \textbf{End Week} \\
\hline
Project Initiation and Planning & 4 weeks & 1 & 4 \\
\hline
Requirements and Analysis & 8 weeks & 5 & 12 \\
\hline
System Design & 4 weeks & 9 & 12 \\
\hline
Development & 12 weeks & 13 & 24 \\
\hline
Testing and QA & 4 weeks & 25 & 28 \\
\hline
Deployment and Training & 3 weeks & 29 & 31 \\
\hline
Go-Live and Support & 3 weeks & 32 & 34 \\
\hline
Project Closure & 2 weeks & 35 & 36 \\
\hline
\textbf{Total} & \textbf{36 weeks} & \textbf{1} & \textbf{36} \\
\hline
\end{tabular}
\end{table}

% ============================================================================
\section{Schedule Assumptions and Constraints}

\subsection{Assumptions}

\begin{itemize}[leftmargin=*]
    \item Resources available as planned throughout project duration
    \item No major holidays or institutional closures affecting schedule
    \item Stakeholder decisions made within agreed timeframes
    \item No significant scope changes after design approval
    \item Testing environment available when needed
    \item Production environment ready for deployment as scheduled
\end{itemize}

\subsection{Constraints}

\begin{itemize}[leftmargin=*]
    \item \textbf{Hard Deadline:} System must be operational before Week 32 (registration period)
    \item \textbf{Resource Constraints:} Limited number of developers and testers
    \item \textbf{External Dependencies:} Integration with existing systems requires coordination
    \item \textbf{Regulatory:} Must allow sufficient UAT time per university policies
    \item \textbf{Seasonal:} Summer months may have limited stakeholder availability
\end{itemize}

% ============================================================================
\section{Schedule Risk Management}

\subsection{Schedule Risks}

\begin{enumerate}[leftmargin=*]
    \item \textbf{Requirements Delay:} Late stakeholder approvals could delay design phase
    \item \textbf{Integration Challenges:} Legacy system integration may take longer than estimated
    \item \textbf{Resource Unavailability:} Key personnel absence could impact critical activities
    \item \textbf{Testing Issues:} Discovery of major defects could extend testing phase
\end{enumerate}

\subsection{Schedule Buffer}

\begin{itemize}[leftmargin=*]
    \item 2-week buffer built into overall schedule
    \item Critical path activities have priority for buffer allocation
    \item Buffer consumption monitored and reported weekly
    \item Early warning system for schedule variance exceeding ±3 days
\end{itemize}

% ============================================================================
\section{Schedule Control Procedures}

\subsection{Monitoring and Reporting}

\begin{itemize}[leftmargin=*]
    \item Weekly schedule status updates
    \item Bi-weekly critical path analysis
    \item Monthly milestone tracking reports
    \item Immediate escalation if milestone at risk
\end{itemize}

\subsection{Change Control}

\begin{itemize}[leftmargin=*]
    \item All schedule changes require impact analysis
    \item Changes affecting milestones require sponsor approval
    \item Schedule baseline updates documented and communicated
    \item Version control maintained for all schedule revisions
\end{itemize}
