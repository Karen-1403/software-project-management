% ============================================================================
% CHAPTER 6: GANTT CHART AND MILESTONES
% ============================================================================

\chapter{Gantt Chart and Milestones}

\vspace{12pt}

\noindent This chapter presents the project schedule in the form of a Gantt chart, showing all project activities, their durations, dependencies, and relationships. Additionally, all project milestones are identified and justified.

% ============================================================================
\section{Project Schedule Overview}

\par The project schedule spans 28 weeksfrom initiation to closure. The Gantt chart provides a visual timeline of all activities, showing start and end dates, durations, and dependencies between tasks.


\section{Gantt Chart}

\begin{landscape}

\begin{figure}[H]
    \centering
    % Insert Gantt chart created in MS Project
    % Export from MS Project as high-quality image or PDF
    % \includegraphics[width=\linewidth]{images/gantt-chart.pdf}
    \caption{Project Gantt Chart - Complete Schedule}
    \label{fig:gantt-chart}
\end{figure}

\par \textit{Note: Create a comprehensive Gantt chart in Microsoft Project showing:}


\end{landscape}

% ============================================================================
\section{Project Milestones}

\par Milestones are significant events or decision points in the project that have zero duration but mark important achievements. The following table lists all project milestones with their justifications.

\begin{table}[H]
\centering
\small
\caption{Project Milestones and Justifications}
\begin{tabularx}{\textwidth}{|c|l|l|X|}
\hline
\textbf{No.} & \textbf{Milestone} & \textbf{Target Date} & \textbf{Justification} \\
\hline
M1 & Project Planning Approved & Week 4 & Ensures all stakeholders agree on project scope, timeline, and budget before commencing work. \\
\hline
M2 & Requirements Approved & Week 8 & Confirms all requirements are documented, validated, and approved. \\
\hline
M3 & Design Approved & Week 12 & Validates technical architecture before development begins. \\
\hline
M4 & Backend Complete & Week 18 & Marks completion of core business logic. Critical for frontend integration. \\
\hline
M5 & Frontend Complete & Week 21 & Signifies completion of user interface. Enables integration testing. \\
\hline
M6 & Integration Complete & Week 23 & Confirms all system components work together seamlessly. \\
\hline
M7 & Development Complete & Week 24 & Code freeze and transition to formal testing phase. \\
\hline
M8 & System Testing Complete & Week 27 & All tests passed. System meets quality standards. \\
\hline
M9 & UAT Approved & Week 28 & Stakeholder validation that system meets business needs. \\
\hline
M10 & System Deployed & Week 30 & Production deployment successful and verified. \\
\hline
M11 & Training Complete & Week 31 & All users trained and prepared to use the system. \\
\hline
M12 & System Go-Live & Week 32 & System officially operational for all students. \\
\hline
M13 & First Registration Complete & Week 34 & Successful completion of first registration cycle. \\
\hline
M14 & Project Closed & Week 36 & All deliverables accepted, documentation complete. \\
\hline
\end{tabularx}
\end{table}

% ============================================================================
\section{Milestone Importance and Impact}

\subsection{Critical Milestones}

\par The following milestones are critical to project success and have the highest priority:

\begin{enumerate}[leftmargin=*]
    \item \textbf{M3 - Design Approved:} Prevents costly rework during development by ensuring design correctness upfront
    \item \textbf{M9 - UAT Approved:} Final validation before production deployment; failure requires major remediation
    \item \textbf{M12 - System Go-Live:} Time-sensitive milestone aligned with academic calendar; delay impacts entire institution
\end{enumerate}

\subsection{Milestone Dependencies}

\par Each milestone has specific dependencies and enables subsequent project work:

\begin{table}[H]
\centering
\caption{Milestone Dependencies}
\begin{tabularx}{\textwidth}{|l|X|X|}
\hline
\textbf{Milestone} & \textbf{Prerequisites} & \textbf{Enables} \\
\hline
M1 & Stakeholder approval, resource allocation & Requirements gathering activities \\
\hline
M2 & Complete requirements documentation & Design activities, effort estimation \\
\hline
M3 & Approved requirements, design reviews & Development work to commence \\
\hline
M7 & All development work packages complete & Formal testing phase \\
\hline
M9 & Successful testing, stakeholder UAT & Production deployment authorization \\
\hline
M12 & Deployment, training, verification & Student registration activities \\
\hline
\end{tabularx}
\end{table}

\section{Key Activity Durations Summary}

\begin{table}[H]
\centering
\caption{Major Phase Durations}
\begin{tabular}{|l|c|c|c|}
\hline
\textbf{Phase} & \textbf{Duration} & \textbf{Start Week} & \textbf{End Week} \\
\hline
Project Initiation and Planning & 3 weeks & 1 & 3 \\
\hline
Requirements Analysis and Design & 3 weeks & 4 & 6 \\
\hline
Development and Implementation & 10 weeks & 7 & 16 \\
\hline
Testing and Quality Assurance & 5 weeks & 17 & 21 \\
\hline
Deployment and Training & 5 weeks & 22 & 26 \\
\hline
Project Closure and Handover & 2 weeks & 27 & 28 \\
\hline
\textbf{Total} & \textbf{28 weeks} & \textbf{1} & \textbf{28} \\
\hline
\end{tabular}
\end{table}

% ============================================================================
\section{Schedule Assumptions and Constraints}

\subsection{Assumptions}

\begin{itemize}[leftmargin=*]
    \item Resources available as planned throughout project duration
    \item No major holidays or institutional closures affecting schedule
    \item Stakeholder decisions made within agreed timeframes
    \item No significant scope changes after design approval
    \item Testing environment available when needed
    \item Production environment ready for deployment as scheduled
\end{itemize}

\subsection{Constraints}

\begin{itemize}[leftmargin=*]
    \item \textbf{Hard Deadline:} System must be operational before Week 30 (registration period)
    \item \textbf{Resource Constraints:} Limited number of developers and testers
    \item \textbf{External Dependencies:} Integration with existing systems requires coordination
    \item \textbf{Regulatory:} Must allow sufficient UAT time per university policies
    \item \textbf{Seasonal:} Summer months may have limited stakeholder availability
\end{itemize}

% ============================================================================
\section{Schedule Risk Management}

\subsection{Schedule Risks}

\begin{enumerate}[leftmargin=*]
    \item \textbf{Requirements Delay:} Late stakeholder approvals could delay design phase
    \item \textbf{Integration Challenges:} Legacy system integration may take longer than estimated
    \item \textbf{Resource Unavailability:} Key personnel absence could impact critical activities
    \item \textbf{Testing Issues:} Discovery of major defects could extend testing phase
\end{enumerate}

% ============================================================================
\section{Schedule Control Procedures}

\subsection{Monitoring and Reporting}

\begin{itemize}[leftmargin=*]
    \item Weekly schedule status updates
    \item Bi-weekly critical path analysis
    \item Monthly milestone tracking reports
\end{itemize}

\subsection{Change Control}

\begin{itemize}[leftmargin=*]
    \item All schedule changes require impact analysis
    \item Changes affecting milestones require sponsor approval
    \item Schedule baseline updates documented and communicated
\end{itemize}
