% ============================================================================
% CHAPTER 14: SCHEDULE ANALYSIS (ES, EF, LS, LF, FLOAT)
% ============================================================================

\chapter{Schedule Analysis: ES, EF, LS, LF, and Float}

\vspace{12pt}

\noindent This chapter presents detailed schedule analysis calculations including Early Start (ES), Early Finish (EF), Late Start (LS), Late Finish (LF), Free Float, and Total Float for all project activities.

% ============================================================================
\section{Schedule Calculation Overview}

\subsection{Definitions}

\begin{description}[leftmargin=3cm, style=nextline]
    \item[ES (Early Start)] The earliest time an activity can start, considering all predecessors
    \item[EF (Early Finish)] ES + Duration (earliest time an activity can finish)
    \item[LS (Late Start)] Latest time an activity can start without delaying the project
    \item[LF (Late Finish)] LS + Duration (latest time an activity can finish without project delay)
    \item[Total Float] LF - EF or LS - ES (maximum delay without delaying project)
    \item[Free Float] Minimum ES of successors - EF (delay without delaying successors)
\end{description}

\subsection{Calculation Methods}

\begin{itemize}[leftmargin=*]
    \item \textbf{Forward Pass:} Calculate ES and EF from project start to finish
    \item \textbf{Backward Pass:} Calculate LS and LF from project finish to start
    \item \textbf{Float Calculations:} Determine scheduling flexibility for each activity
\end{itemize}

% ============================================================================
\section{Complete Schedule Analysis Table}

\begin{landscape}

\begin{table}[H]
\centering
\caption{Complete Schedule Analysis with ES, EF, LS, LF, and Float}
\scriptsize
\begin{tabularx}{\linewidth}{|l|X|c|l|c|c|c|c|c|c|l|}
\hline
\textbf{ID} & \textbf{Activity} & \textbf{Dur.} & \textbf{Predecessors} & \textbf{ES} & \textbf{EF} & \textbf{LS} & \textbf{LF} & \textbf{TF} & \textbf{FF} & \textbf{Critical?} \\
 & & \textbf{(wks)} & & & & & & & & \\
\hline
A & Project Initiation & 2 & - & 0 & 2 & 0 & 2 & 0 & 0 & Yes \\
\hline
B & Requirements Gathering & 4 & A & 2 & 6 & 2 & 6 & 0 & 0 & Yes \\
\hline
C & Requirements Documentation & 2 & B & 6 & 8 & 6 & 8 & 0 & 0 & Yes \\
\hline
D & System Architecture Design & 3 & C & 8 & 11 & 8 & 11 & 0 & 0 & Yes \\
\hline
E & Database Design & 2 & C & 8 & 10 & 9 & 11 & 1 & 1 & No \\
\hline
F & UI/UX Design & 3 & C & 8 & 11 & 9 & 12 & 1 & 1 & No \\
\hline
G & Backend Dev Setup & 1 & D, E & 11 & 12 & 11 & 12 & 0 & 0 & Yes \\
\hline
H & Prerequisite Engine Dev & 3 & G & 12 & 15 & 12 & 15 & 0 & 0 & Yes \\
\hline
I & Conflict Detection Module & 2 & G & 12 & 14 & 13 & 15 & 1 & 1 & No \\
\hline
J & Waitlist Management & 2 & G & 12 & 14 & 13 & 15 & 1 & 1 & No \\
\hline
K & Registration Workflow & 3 & H, I, J & 15 & 18 & 15 & 18 & 0 & 0 & Yes \\
\hline
L & Frontend Dev Setup & 1 & F & 11 & 12 & 12 & 13 & 1 & 1 & No \\
\hline
M & Student Portal Dev & 4 & L & 12 & 16 & 13 & 17 & 1 & 1 & No \\
\hline
N & Admin Dashboard Dev & 3 & L & 12 & 15 & 14 & 17 & 2 & 2 & No \\
\hline
O & Course Catalog Interface & 2 & L & 12 & 14 & 15 & 17 & 3 & 0 & No \\
\hline
P & Timetable Builder & 3 & O & 14 & 17 & 15 & 18 & 1 & 1 & No \\
\hline
Q & Frontend-Backend Integration & 2 & K, M, N, P & 18 & 20 & 18 & 20 & 0 & 0 & Yes \\
\hline
R & SIS Integration & 2 & Q & 20 & 22 & 20 & 22 & 0 & 0 & Yes \\
\hline
S & Authentication Integration & 1 & Q & 20 & 21 & 21 & 22 & 1 & 1 & No \\
\hline
T & Email Notification System & 1 & Q & 20 & 21 & 21 & 22 & 1 & 1 & No \\
\hline
U & Unit Testing & 2 & K, Q & 20 & 22 & 20 & 22 & 0 & 0 & No* \\
\hline
V & Integration Testing & 2 & R, S, T, U & 22 & 24 & 22 & 24 & 0 & 0 & Yes \\
\hline
W & System Testing & 2 & V & 24 & 26 & 24 & 26 & 0 & 0 & Yes \\
\hline
X & Performance Testing & 1 & W & 26 & 27 & 27 & 28 & 1 & 1 & No \\
\hline
Y & Security Testing & 1 & W & 26 & 27 & 27 & 28 & 1 & 1 & No \\
\hline
Z & User Acceptance Testing & 2 & W, X, Y & 27 & 29 & 27 & 29 & 0 & 0 & Yes \\
\hline
AA & Deployment Planning & 1 & Z & 29 & 30 & 29 & 30 & 0 & 0 & Yes \\
\hline
AB & Production Deployment & 1 & AA & 30 & 31 & 30 & 31 & 0 & 0 & Yes \\
\hline
AC & Training Material Dev & 2 & W & 26 & 28 & 29 & 31 & 3 & 3 & No \\
\hline
AD & Training Delivery & 1 & AB, AC & 31 & 32 & 31 & 32 & 0 & 0 & Yes \\
\hline
AE & Go-Live Support & 2 & AD & 32 & 34 & 32 & 34 & 0 & 0 & Yes \\
\hline
AF & Project Closure & 1 & AE & 34 & 35 & 34 & 35 & 0 & 0 & Yes \\
\hline
\end{tabularx}
\end{table}

\par \textit{*U is on a parallel critical path with zero float}

\end{landscape}

% ============================================================================
\section{Critical Path Activities}

\par Activities with Total Float = 0 form the critical path(s):

\begin{center}
\textbf{Primary Critical Path:} \\
A → B → C → D → G → H → K → Q → R → V → W → Z → AA → AB → AD → AE → AF \\
\vspace{0.5cm}
\textbf{Parallel Critical Path:} \\
A → B → C → D → G → H → K → Q → U → V → W → Z → AA → AB → AD → AE → AF
\end{center}

\par \textbf{Total Project Duration: 35 weeks}

% ============================================================================
\section{Float Analysis}

\subsection{Activities with Total Float > 0}

\begin{table}[H]
\centering
\caption{Non-Critical Activities with Scheduling Flexibility}
\begin{tabular}{|l|l|c|l|}
\hline
\textbf{Activity ID} & \textbf{Activity Name} & \textbf{Total Float} & \textbf{Implication} \\
 & & \textbf{(weeks)} & \\
\hline
AC & Training Material Development & 3 & Can be delayed without impact \\
\hline
O & Course Catalog Interface & 3 & Low priority for resources \\
\hline
N & Admin Dashboard Development & 2 & Some flexibility \\
\hline
E & Database Design & 1 & Near-critical, monitor closely \\
\hline
F & UI/UX Design & 1 & Near-critical, monitor closely \\
\hline
I & Conflict Detection Module & 1 & Near-critical, monitor closely \\
\hline
J & Waitlist Management & 1 & Near-critical, monitor closely \\
\hline
L & Frontend Dev Setup & 1 & Near-critical, monitor closely \\
\hline
M & Student Portal Development & 1 & Near-critical, monitor closely \\
\hline
P & Timetable Builder & 1 & Near-critical, monitor closely \\
\hline
S & Authentication Integration & 1 & Near-critical, monitor closely \\
\hline
T & Email Notification System & 1 & Near-critical, monitor closely \\
\hline
X & Performance Testing & 1 & Near-critical, monitor closely \\
\hline
Y & Security Testing & 1 & Near-critical, monitor closely \\
\hline
\end{tabular}
\end{table}

\subsection{Near-Critical Activities}

\par Activities with Total Float ≤ 1 week are considered near-critical and require close monitoring:

\begin{itemize}[leftmargin=*]
    \item Any delay in these activities could make them critical
    \item Resource prioritization should favor these after critical activities
    \item Weekly status monitoring recommended
    \item Consider adding buffer resources if delays occur
\end{itemize}

% ============================================================================
\section{Schedule Compression Opportunities}

\subsection{Fast-Tracking Opportunities}

\par Activities that could potentially overlap:

\begin{enumerate}[leftmargin=*]
    \item Database Design (E) could start during Requirements Documentation (C) - saves 1 week
    \item UI/UX Design (F) could start during Requirements Documentation (C) - saves 1 week
    \item Training Material Development (AC) has 3 weeks float - can be done earlier
\end{enumerate}

\subsection{Crashing Opportunities}

\par Critical activities that could be shortened by adding resources:

\begin{table}[H]
\centering
\caption{Crashing Analysis for Critical Activities}
\begin{tabular}{|l|c|c|c|r|}
\hline
\textbf{Activity} & \textbf{Normal} & \textbf{Crashed} & \textbf{Time} & \textbf{Cost} \\
 & \textbf{Duration} & \textbf{Duration} & \textbf{Saved} & \textbf{Increase} \\
\hline
H (Prerequisite Engine) & 3 weeks & 2 weeks & 1 week & 50,000 EGP \\
\hline
K (Registration Workflow) & 3 weeks & 2 weeks & 1 week & 50,000 EGP \\
\hline
M (Student Portal) & 4 weeks & 3 weeks & 1 week & 40,000 EGP \\
\hline
W (System Testing) & 2 weeks & 1.5 weeks & 0.5 week & 30,000 EGP \\
\hline
\end{tabularx>
\end{table}

% ============================================================================
\section{Schedule Management Recommendations}

\begin{enumerate}[leftmargin=*]
    \item \textbf{Focus on Critical Path:} Prioritize resources for critical activities (TF = 0)
    \item \textbf{Monitor Near-Critical:} Weekly tracking of activities with TF ≤ 1 week
    \item \textbf{Utilize Float Strategically:} Use non-critical activities as resource buffers
    \item \textbf{Early Warning System:} Alert when critical activities show signs of delay
    \item \textbf{Resource Leveling:} Use float to smooth resource demands
    \item \textbf{Buffer Management:} Protect critical path with schedule buffers
    \item \textbf{Fast-Track Carefully:} Only overlap activities when risks are acceptable
\end{enumerate}
